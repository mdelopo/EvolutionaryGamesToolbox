\section{Introduction}
\subsection{Description of the problem}
This is the report for the project of the course ``Game Theory'' of the Aristotle University of Thessaloniki. The subject of this project is the implementation and study of the Axelrod evolutionary tournament in the game "Prisoner's Dilemma." This name refers to a social dilemma, that is, any game of the form $\begin{bmatrix} R & S \\ T & P \end{bmatrix}$, where the following condition holds: $S < P < R < T$. Each player is given a choice, whether to cooperate or to defect. The payoff is calculated based on the player's move as well as the opponent's move; if both players cooperate, they each get R, if one cooperates and the other defects, the defector receives T and the cooperator receives S, otherwise if they both defect they both get rewarded P points. Paradoxically, a rational player may notice that no matter what the opponent plays, it is always better to defect, thus creating a Nash equilibrium (a game outcome that no player has motive to move away from) for two perfectly rational players in the ``Defect-Defect'' zone, making them both miss out on possible extra points, were they to cooperate. 

The repeated iteration of this game creates a match between strategies; multiple matches between strategies form a tournament, and the computation of a new population based on each strategy’s performance in the tournament constitutes the main focus of the study, an evolutionary tournament.

The project consists of four main functions, each of which performs a specific task that can be characterized by two features of the respective assumption: the nature of the simulation (theoretical or real) and the evolutionary dynamic (fitness or imitation). Each function is introduced individually, with an explanation of its operation and any assumptions made in each case. The first part of the project is largely based on the study by Mathieu et al. In fact, an effort was made to replicate the results as closely as possible to those presented in their 1999 paper.

For the study of the project, some concepts are crucial to understand. More specifically:
\begin{enumerate}
  \item A match is the base game played between two players, with a specified number of rounds (amount of times the players choose a move) which is unknown to the players.
  \item A strategy is a specific algorithm followed by a player in order to calculate their next move in a match.
  \item A population is the number of players following each strategy in a given tournament.
  \item A tournament consists of all possible matches between each pair of players (a player does not play against themselves though).
  \item An evolutionary game is the repetition of a tournament for a specified number of generations, where the population of each generation is calculated based on specified evolutionary dynamics.
\end{enumerate}

As mentioned above, the evolution dynamics studied in this project are the following:
\begin{enumerate}
  \item Fitness dynamics, where for each generation the number of players adopting each strategy is proportional to the performance (measured by a specific metric, in this case the total payoff in the tournament) of that strategy.
  \item Imitation dynamics, where after each generation a specified number of players adopt the best strategy (calculated either as an Individual or as a Total, more on that on a later chapter) of the previous generation.
\end{enumerate}

\subsection{Related Work}
The subject of the project was in large popularized by the famous Axelrod Tournament\cite{axelrod1981}\cite{axelrod1988}, where many famous game theorists were challenged to submit strategies for the Iterated Prisoner's Dilemma. The winning strategy ended up being the well-known Tit For Tat strategy, which, despite its simpleness, had many redeeming qualities; it was nice, meaning it was never the first to defect. It was retaliatory, meaning it was ready to counter attack if the opponent defects. It was also forgiving, meaning it did not hold a grudge too long; if an opponent defected but repented, Tit For Tat forgave them and continued to cooperate. Lastly, it was clear; it is not difficult for the opponent to understand the intentions of the strategy, thus making it more likely that cooperation occurs.

Since 1980, when the tournament was held, a large number of studies have been published on the matter, a lot of which have focused on the evolutionary aspect of the tournament, trying to mimic the actual evolution of species in the animal kingdom, as well as the birth of cooperation between humans. One such paper was published in 1999 by Philippe Mathieu, Bruno Beaufils and Jean-Paul Delahaye with title ``Studies on Dynamics in the Classical Iterated Prisoner's Dilemma with Few Strategies''\cite{mathieu1999}. In this specific paper, the evolutionary dynamics analyzed were Fitness Dynamics and the results aimed to showcase the different possible forms of dynamics that can occur and which aspects of the simulation they may depend on. As mentioned previously, the first part of this project is based exactly on that paper, aiming to recreate the results presented by Mathieu et al, as well as modifying the dynamics slightly to notice any differences in the resulting dynamics.